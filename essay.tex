\documentclass{article}
\usepackage[english]{babel}
\usepackage[letterpaper,top=2cm,bottom=2cm,left=3cm,right=3cm,marginparwidth=1.75cm]{geometry}
\usepackage{amsmath,amssymb,amsthm,amsopn,  float}
\usepackage{tikz}
\usepackage{tikz-cd}
\usepackage{physics}
\usepackage{graphicx}
\usepackage[colorlinks=true, allcolors=blue]{hyperref}
\usepackage{longtable,lipsum} 


\newtheorem{theorem}{Theorem}
\newtheorem{lemma}[theorem]{Lemma} % Lemmas share numbering with theorems
\newtheorem{corollary}[theorem]{Corollary}
\theoremstyle{definition}
\newtheorem{definition}[theorem]{Definition}
\newtheorem{example}[theorem]{Example}


\title{Symlectic Geometry}
\author{Radim Čech}

\begin{document}
\maketitle

\noindent Symplectic geometry is a branch of differential geometry that studies symplectic manifolds, which are smooth manifolds equipped with a closed, non-degenerate 2-form called a symplectic form. It originated from classical mechanics.

\begin{definition}[Symlectic manifold]
    Let $M$ be a smooth manifold of even dimension $2m$ and let $\omega \in \Omega^2(M)$ be a closed non degenerate 2-form i.e.
    \begin{equation*}
        d\omega = 0 \text{ and } \omega^m = \omega \wedge \omega \wedge \dots \wedge \omega \not = 0,
    \end{equation*}
    Then $\omega$ is called a \textit{simplectic form} and the pair $(M, \omega)$ is called a \textit{simplectic manifold}.
\end{definition}

ekvivalentni definice nedegenerovanosti.

Narozdil od riemannovske geometrie nelze pouzit partitions of unity na konstrukci metriky.

napsat poznamku o koncenci se psanim dimenze manifoldu :D


\begin{example}[Canonical symplectic structure]
    Let $M = \mathbb{R}^{2m}$ with the global coordinates ${q_1, \dots, q_m, p_1, \dots, p_m}$. and let $\omega$ be a form s.t., 
    \begin{equation*}
        \omega = \sum_{i=1}^m dp_i \wedge dq_i.
    \end{equation*}
    Then 
    \begin{equation*}
        \omega^m = m! \cdot (-1)^{m(m-1)/2} \cdot dp_1 \wedge \dots \wedge dp_m \wedge dq_1 \wedge \dots \wedge dq_m.
    \end{equation*}
    We call $R^2m$ with the form $\omega$ the canonical symplectic structure.
\end{example}

\begin{example}[Cotangent bundle is a symplectic manifold.]
    Let $Q$ be a manifold, and consider the manifold $M = T^*Q$. Then there is a canonical 1-form $\theta \in \Omega^1(M)$ given by 
    \begin{equation}\label{eqTheta1}
        \theta(X) =  \langle \pi_{T^*Q}(X) , T(\pi_Q)(X) \rangle , \quad X \in T(T^*Q),
    \end{equation}
    where $\langle \cdot , \cdot \rangle$ is the natural pairing between tangent and cotangent spaces and the projections are the following:
    \[
    \begin{tikzcd}
        & T(T^*Q) \arrow[ld, "\pi_{T^*Q}"'] \arrow[rd, "T(\pi_Q)"] & \\
        T^*Q \arrow[rd, "\pi_Q"'] & & TQ \arrow[ld, "\pi_Q"] \\
        & Q &
        \end{tikzcd}
        \quad
        \begin{tikzcd}
        & (q, p; \xi, \eta) \arrow[ld, "\pi_{T^*Q}"'] \arrow[rd, "T(\pi_Q)"] & \\
        (q, p) \arrow[rd, "\pi_Q"'] & & (q, \xi) \arrow[ld, "\pi_Q"] \\
        & q &
    \end{tikzcd}
    \]    
    Let $ q = (q^1, \ldots, q^n) : U \to \mathbb{R}^n$ be a chart on $Q$, the we have the induced chart $T^*q : T^*U \to \mathbb{R}^n \times \mathbb{R}^n$, where $T^*_x q = (T_x q^{-1})^*$, we put $p_i := \langle e_i , T^*q(\cdot) \rangle : T^*U \to \mathbb{R}$. Then $(q^1, \ldots, q^n, p_1, \ldots, p_n) : T^*U \to \mathbb{R}^n \times (\mathbb{R}^n)^*$ are the induced coordinates and locally in these coordinates 
    \begin{equation}\label{eqTheta2}
        \theta(q,p) = \sum_{i=1}^n \left( \theta \left( \frac{\partial}{\partial q^i} \right) dq^i + \theta \left( \frac{\partial}{\partial p_i} \right) dp_i \right) = \sum_{i=1}^n p_i \, dq^i + 0,
    \end{equation}
    since $\theta \left( \frac{\partial}{\partial q^i} \right) = \theta_{R^n} \left( (q,p;e_i,0) \right) = \langle p,e_i \rangle = p_i$.

    Now we define the 2-form $\omega \in \Omega^2(T^*Q)$ by 
    \begin{equation}\label{eqCotangentForm}
        \omega := -d \theta \stackrel{\text { locally }}{=} \sum_{i=1}^n d q^i \wedge d p_i.
    \end{equation}
    We see that the 2-form $\omega$ is non-degenerate.
\end{example}

\begin{definition}
    The form $\theta \in \Omega^1(M)$ from \eqref{eqTheta1}, locally given by \eqref{eqTheta2}, is called the \textit{tautological 1-form} on $T^*Q$. The induced 2-form $\omega$ from \eqref{eqCotangentForm} is called the \textit{canonical symplectic structure} on $T^*Q$.
\end{definition}

dukaz ze je neni degen?

\begin{definition}
    Let $X: J \times M \rightarrow TM$ be a smooth mapping such that $\pi_M \circ X = pr_2$, where $J$ is open. Then we call $X$ a \textit{time dependent vector field} on a manifold $M$. 
\end{definition}
There is an associated vector field $\bar{X} \in \mathfrak{X}(J \times M)$, given by $\bar{X}(t, x)=(\frac{\partial}{\partial t}, X(t,x)) \in T_t\mathbb{R}\times T_xM$.

\begin{definition} \label{defEvolutionOperator}
    Let $X$ be a time dependent vector field on a manifold $M$ and let $\Phi^X: J \times J \times M \rightarrow M$ be a map defined on a maximal neighborhood of $\Delta_J\times M$ satisfying the differential equation
    \begin{equation}
        \begin{split}
            \frac{d}{d t} \Phi^X(t, s, x)&=X\left(t, \Phi^X(t, s, x)\right) \\ 
            \Phi^X(s,s,x) &= x
        \end{split}
    \end{equation}
\end{definition}
Definition \ref{defEvolutionOperator} is equivalent with 
\begin{equation*}
    (t,\Phi^X(t,s,x)) = Fl^{\bar{X}}(t-s, (s,x)),
\end{equation*}
so the evolution operator exits and is unique on a maximal integral curve and satisfies
\begin{equation*}
    \Phi_{t, s}^X=\Phi_{t, r}^X \circ \Phi_{r, s}^X \text{, where }\Phi_{t, r}^X(x) = \Phi(t,s,x).
\end{equation*}

\begin{lemma} \label{lemmaFlows}
    Let $f_t$ be a curve of diffeomorphisms on a manifold $M$ locally defined for each $t$ such that $f_0 = Id$. Defined two time dependent vector fields
    \begin{equation}
        \xi_t(x):=\left(T_x f_t\right)^{-1} \frac{\partial}{\partial t} f_t(x), \hspace{5pt} \eta_t(x):=\left(\frac{\partial}{\partial t} f_t\right)\left(f_t^{-1}(x)\right)
    \end{equation}
    Then $T\left(f_t\right) \cdot \xi_t=\frac{\partial}{\partial t} f_t=\eta_t \circ f_t$, so $\xi_t$ and $\eta_t$ are $f_t$-related. Let $\omega \in \Omega^k(M)$. Then
    \begin{equation}\label{eqFlows1}
        i_{\xi_t} f_t^* \omega=f_t^* i_{\eta_t} \omega, 
    \end{equation}
    \begin{equation}\label{eqFlows2}
        \frac{\partial}{\partial t} f_t^* \omega=f_t^* \mathcal{L}_{\eta_t} \omega=\mathcal{L}_{\xi_t} f_t^* \omega.
    \end{equation}
\end{lemma}

\begin{proof}
    \begin{equation*}
        \begin{split}
           \left(i_{\xi_t} f_t^* \omega\right)_x\left(X_2, \ldots, X_k\right) &= \left(f_t^* \omega\right)_x\left(\xi_t(x), X_2, \ldots, X_k\right)\\
            &=\omega_{f_t(x)}\left(T_x f_t \cdot \xi_t(x), T_x f_t \cdot X_2, \ldots, T_x f_t \cdot X_k\right) \\
            &=\omega_{f_t(x)}\left(\eta_t\left(f_t(x)\right), T_x f_t \cdot X_2, \ldots, T_x f_t \cdot X_k\right) \\
            &= \left(f_t^* i_{\xi_t}  \omega\right)_x\left(X_2, \ldots, X_k\right)
        \end{split}
    \end{equation*}
    This proves \eqref{eqFlows1}. Now consider $\bar{\eta} \in \mathfrak{X}(\mathbb{R} \times M), \bar{\eta}(t, x)=\left(\partial_t, \eta_t(x)\right)$ and let $\Phi^\eta : \mathbb{R} \times \mathbb{R} \times M \rightarrow M$ be the evolution operator, i.e.
    \begin{equation*}
        \frac{\partial}{\partial t} \Phi_{t, s}^\eta(x)=\eta_t\left(\Phi_{t, s}^\eta(x)\right), \quad \Phi_{s, s}^\eta(x) = x,
    \end{equation*}
    such that 
    \begin{equation*}
        \left(t, \Phi_{t, s}^\eta(x)\right)=\mathrm{Fl}_{t-s}^{\bar{\eta}}(s, x), \hspace{4pt} \Phi_{t, s}^\eta=\Phi_{t, r}^\eta \circ \Phi_{r, s}^\eta(x).
    \end{equation*}
    Since $f_t$ satisfies 
    $\frac{\partial}{\partial t} f_t = \eta_t \circ f_t$ and $ f_0 = Id_M$, either $f_t = \Phi^\eta_{t,0}$, or $(t,f_t(x)) = Fl^{\bar{\eta}}_t(0,x)$, so $f_t = pr_2 \circ Fl^{\bar{\eta}}_t \circ ins_0$. Thus
    \begin{equation*}
        \frac{\partial}{\partial t} f_t^* \omega=\frac{\partial}{\partial t}\left(\mathrm{pr}_2 \circ Fl_t^{\bar{\eta}} \circ ins_0\right)^*\omega = ins_0^* \frac{\partial}{\partial t}(Fl_t^{\bar{\eta}})^*pr_2^*\omega = ins_0^*(Fl^{\bar{\eta}}_t)^*\mathbb{L}_{\bar{\eta}}pr_2^*\omega.
    \end{equation*}
    For time dependant vector fields $X_i$(tady mozna nejaka vlastnost lie derivative!!!) we have 
    \begin{equation*}
        \begin{split}
            \left(\mathcal{L}_{\bar{\eta}} \operatorname{pr}_2^* \omega\right) (0 \times X_1, \ldots,0 \times X_k)|_{(t,x)} 
            &= \bar{\eta} (( \operatorname{pr}_2^* \omega) (0 \times X_1, \ldots))|_{(t,x)} \\
            &-\sum_i (\operatorname{pr}_2^* \omega) (0 \times X_1, \ldots, [\bar{\eta}, 0 \times X_i], \ldots, 0 \times X_k)|_{(t,x)}\\
            &=\left(\partial_t, \eta_t(x)\right)\left(\omega\left(X_1, \ldots, X_k\right)\right) - \sum_i \omega\left(X_1, \ldots,\left[\eta_t, X_i\right], \ldots, X_k\right)|_x \\ 
            &= \left(\mathcal{L}_{\eta_t} \omega\right)_x\left(X_1, \ldots, X_k\right).
        \end{split}
    \end{equation*}
    For $X_i \in T_xM$, this implies
    \begin{equation}
        \begin{split}
            \left( \frac{\partial}{\partial t} f_t^* \omega \right)_x (X_1, \ldots, X_k) 
            &= \left( \operatorname{ins}^*\bigl(\mathrm{Fl}^{\eta}_t\bigr)^* \mathcal{L}_\eta \, \mathrm{pr}_2^* \omega \right)_x (X_1, \ldots, X_k) \\
            &= \left( \left(\mathrm{Fl}^{\eta}_t\right)^* \mathcal{L}_\eta \, \mathrm{pr}_2^* \omega \right)_{(0,x)} (0 \times X_1, \ldots, 0 \times X_k) \\
            &= \left( \mathcal{L}_\eta \, \mathrm{pr}_2^* \omega \right)_{(t, f_t(x))} (0_t \times T_x f_t \cdot X_1, \ldots, 0_t \times T_x f_t \cdot X_k) \\
            &= \left( \mathcal{L}_{\eta_t} \omega \right)_{f_t(x)} (T_x f_t \cdot X_1, \ldots, T_x f_t \cdot X_k) \\
            &= \left( f_t^* \mathcal{L}_{\eta_t} \omega \right)_x (X_1, \ldots, X_k),
        \end{split}
    \end{equation}
    We have proven the first part of \eqref{eqFlows2}, the second part follow from \eqref{eqFlows1}
    \begin{equation}
        \begin{split}
        \frac{\partial}{\partial t} f_t^* \omega 
        &= f_t^* \mathcal{L}_{\eta_t} \omega \\
        &= f_t^* \left( d i_{\eta_t} + i_{\eta_t} d \right) \omega \\
        &= d f_t^* i_{\eta_t} \omega + f_t^* i_{\eta_t} d \omega \\
        &= d i_{\xi_t} f_t^* \omega + i_{\xi_t} f_t^* d \omega \\
        &= d i_{\xi_t} f_t^* \omega + i_{\xi_t} d f_t^* \omega \\
        &= \mathcal{L}_{\xi_t} f_t^* \omega.
        \end{split}
        \end{equation}
\end{proof}

dopsat co je ins a pr2?

\begin{theorem}[Darboux]
    Let $(M,\omega)$ be a symplectic manifold of dimension $2n$. Then for all points $x \in M$ exists a chart $(U,u)$ centered at $x$ such that $\omega |_U = \sum_{i=1}^{n} du^i \wedge du^{n+i}$.
\end{theorem}

\begin{proof}
    Take a chart $(U,u)$ centered at $x$ and choose coordinates such that $\omega_x = \sum_{i=1}^{n} du^i \wedge du^{n+i}$ at $x$. Then $\omega_0 = \omega |_U$ and $\omega_1 = \sum_{i=1}^{n} du^i \wedge du^{n+i}$ are two symplectic forms that are equal at $x$. Now interpolate $\omega_t = \omega_0 + t(\omega_1 + \omega_0)$. Then $\omega_t$ is a symplectic form on a possibly smaller neighbourhood of $x$ for all $t \in [0,1]$.

    We want to find a curve of diffeomorphisms $f_t$ near $x$ such that $f_0 = id$, $f_t(x) = x$ and such that the pullback condition $f^*_t \omega_t = \omega_0$ is satisfied. Assume that $U$ is contractible, then the second cohomology group $H^2(U) = 0$ and every closed 2-form is exact, so $d(\omega_1 - \omega_0) = 0$ implies $\omega_1-\omega_0 = d \psi$ for some $\psi \in \Omega^1(U)$. By adding a constant we may assume that $\psi_x = 0$. Now by using Lemma \ref{lemmaFlows}, \eqref{eqFlows2}, we get a time dependant vector field $\eta_t = \frac{\partial}{\partial t}f_t \circ f^{-1}_t$, then by differentiating with respect to $t$, (cartan formula!!)
    \begin{equation*}
        0=\frac{\partial}{\partial t} f_t^* \omega_t=f_t^*\left(\mathcal{L}_{\eta_t} \omega_t+\frac{\partial}{\partial t} \omega_t \right) = f_t^*\left(d i_{\eta_t} \omega_t+i_{\eta_t} d \omega_t+\omega_1-\omega_0\right)=f_t^* d\left(i_{\eta_t}\omega_t + \psi\right)
    \end{equation*}

    Since $\omega_t$ is non-degenerate, the equation $i_{\eta_t}\omega_t = -\psi$ prescribes the vector field $\eta_t$ uniquely. Also $\eta_t(x) = 0$ sine $\psi_x = 0$. On some neighbourhood of $x$ the left evolution operator $f_t$ of $\eta_t$ exists for all $t \in [0,1]$ and $\frac{\partial}{\partial t}(f^*_t\omega_t) = 0$, so $f_t^*\omega_t = \omega_0$ for all $t \in [0,1]$.
\end{proof}

NOW lets study symplectomorphisms.

mozna dopsat definici lagrangian manifold

\begin{definition}[Lagrangian submanifold]
    Let $(M^{2m},\omega)$ be a symplectic manifold. We call a submanifold $Y$ of $M$ lagrangian, if at each $p \in Y$ , $T_pY$ is a lagrangian
    subspace of $T_pM$ , that is, $\omega_p|_{T_pY} \equiv 0$ and $\operatorname{dim} T_pY = \frac{1}{2} \operatorname{dim} T_pM$.

    Equivalently, if $i : Y \rightarrow M$ is the inclusion map, then $Y$ is lagrangian if and only if $i*\omega = 0$ and $\operatorname{dim} Y = \frac{1}{2} \operatorname{dim} M$ 
    
\end{definition}

\begin{example}[The zero section of $T^*M$]
    Let $M^m$ be a manifold and consider its cotangent bundle $T^*M$ with the local coordinates $x_1, \ldots, x_m, \xi_1, \ldots, \xi_m$ on $T^*U$. Then the \textit{zero section} of $T^*M$ is the set 
    \begin{equation*}
        M_0 = {(x,\xi) \in T^*M | \xi = 0 in T^*_xM}
    \end{equation*}
    is an $m$-dimensional submanifold of $T^*M$ whose intersection with $T*U$ is given by the equations $\xi_1 = \ldots = \xi_n = 0$. Then clearly the tautological 1-form $\theta(x, \xi)$ vanishes on $M_0 \cap T^*U$. Let $i_0 : M_0 \rightarrow T^*M$ be the inclusion map, then $i_0^* \theta = 0$. Hence $i_0^* \omega = i_0^*d \theta = 0$ and so $M_0$ is lagrangian.
\end{example}

Now let $\mu \in \Omega^1(M)$ be a 1-form and consider the set 
\begin{equation}\label{eqNejakySubmanifold}
    M_\mu=\left\{\left(x, \mu_x\right) \mid x \in X, \mu_x \in T_x^* X\right\}
\end{equation}
Then $M_\mu$ is a submanifold of $T^*M$. When is $M_\mu$ lagrangian?

\begin{lemma}
    Let $M_\mu$ be of the form \eqref{eqNejakySubmanifold}. Denote by $s_\mu : M \rightarrow T^*M, x \rightarrow (x, \mu_x)$ the 1-form $\mu$ regarded as a map. Let $\theta$ be the tautological 1-form on $T^*M$. Then
    \begin{equation*}
        s^*_\mu \theta = \mu
    \end{equation*}
\end{lemma}
\begin{proof}
    By definition of $\theta$, $\theta_p = (d\pi)^* \xi$ at $p = (x, \xi) \in M.$
For $p = s_\mu(x) = (x, \mu_x)$, we have $\theta_p = (d\pi_p)^* \mu_x$. Then 

\begin{equation*}
    (s_\mu^* \theta)_x = (ds_\mu)^*_x \theta_p = (ds_\mu)^*_x (d\pi_p)^* \mu_x = \left( d(\pi \circ s_\mu) \right)^*_x \mu_x = \mu_x.
\end{equation*}
\end{proof}

\begin{lemma}
    Let $M_\mu$ be of the form \eqref{eqNejakySubmanifold}. Then $M_\mu$ is lagrangian iff. $\mu$ is closed.
\end{lemma}
\begin{proof}
    The map $s_\mu : M \rightarrow T^*M, x \rightarrow (x, \mu_x)$ is an embedding with image $M_\mu$. Then there is a diffeomorphism \( \tau: M \to M_\mu \), \( \tau(x) := (x, \mu_x) \), such that the following diagram commutes:

    \[
    \begin{tikzcd}
        X \arrow[r, "s_\mu"] \arrow[dr, "\tau"'] & T^*X \\
        & X_\mu \arrow[u, hook, "i"']
    \end{tikzcd}
    \]
    
    We want to express the condition of \( M_\mu \) being Lagrangian in terms of the form \( \mu \):

    \begin{equation}
        \begin{aligned}
        M_\mu \text{ is Lagrangian} 
        &\iff i^* d\theta = 0 \\
        &\iff \tau^* i^* d\theta = 0 \\
        &\iff (i \circ \tau)^* d\theta = 0 \\
        &\iff s_\mu^* d\theta = 0 \\
        &\iff d s_\mu^* \theta = 0 \\
        &\iff d\mu = 0 \\
        &\iff \mu \text{ is closed.}
        \end{aligned}
    \end{equation}

    Therefore, there is a one-to-one correspondence between the set of Lagrangian submanifolds of $T^*M$ of the form \eqref{eqNejakySubmanifold} and the set of closed 1-forms on $M$.
\end{proof}

There are other lagrangian submanifolds of $T^*M$. Lets study the conormal bundles.

\begin{definition}
    Let $S^k$ be submanifold of $M^m$, then the \textit{conormal space} at $x$ is the set
    \begin{equation*}
        N_x^* S=\left\{\xi \in T_x^* M \mid \xi(v)=0\right., \left.v \in T_x M\right\}.
    \end{equation*}
    The \textit{conormal bundle} of $S$ is 
    \begin{equation*}
        N^* S=\left\{(x, \xi) \in T^* M \mid x \in S, \xi \in N_x^* S\right\}.
    \end{equation*}
\end{definition}
\begin{example}
    Let $S \subset X$ be a submanifold, then the conormal bundle $N^*S$ is a lagrangian submanifold of $T^*x$.
\end{example}

\begin{lemma}
    The conormal bundle $N^*S$ is an $n$-dimensional submanifold of $T^*M$.
\end{lemma}

\begin{lemma}
    Let $i: N^*S \hookrightarrow T^*M$ be the inclusion and $\theta$ the tautological 1-form on $T^*M$. Then
    \begin{equation*}
        i^*\theta = 0.
    \end{equation*}
\end{lemma}

\begin{proof}
    Let $(U, x_1, \ldots, x_n)$ be a coordinate system on $M$ centered at $x \in S$ and adapted to $M$, so that $U \cap S$ is described by 
    \begin{equation*}
        x_{k+1} = \ldots = x_n = 0. \text{ and } \xi_1 = \ldots = \xi_k = 0
    \end{equation*}

    Let $(U, x_1, \ldots, x_n, \xi_1, \ldots, \xi_n)$ be the associated cotangent coordinate system. The submanifold $N^*S \cap T^*U$ is then described by
    \begin{equation}
        x_{k+1} = \ldots = x_n = 0 \text{ and } \xi_1 = \ldots = \xi_k = 0.
    \end{equation}

    Since $\theta = \sum \xi_i dx_i$ on $T^*U$, we conclude that, at $p \in N^*S$,
    \begin{equation}
        (i^* \theta)_p = \theta_p|_{T_p(N^*S)} = \left. \sum_{i > k} \xi_i dx_i \right|_{\mathrm{span}\left\{\frac{\partial}{\partial x_i}, i \leq k \right\}} = 0.
    \end{equation}
\end{proof}

\begin{corollary}
    For any submanifold $S$ of $M$, the conormal bundle $N^*S$ is a lagrangian submanifold of $T^*M$.
\end{corollary}

Notice that taking $S = \{x\}$ to be a point, then the conormal bundle $N^*S = T^*_xM$ is a cotangent fiber. Taking $S = X$, the conormal bundle is the zero section $M_0$.


\begin{lemma}[Lemmatko]
$2+2 = 4 - 1 = 3$ quick maffs.
\end{lemma}

A ted si rekneme dulezitou vetu.
\begin{theorem}[Hlavni veta o gaystvi]
Jsi gay.
\end{theorem}


\begin{corollary}
    Vlastne dusledek tohoto kratkeho textu je, ze bych se mel jit zabit. Jdu na to!
\end{corollary}

\end{document}