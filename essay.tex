\documentclass{article}
\usepackage[english]{babel}
\usepackage[letterpaper,top=2cm,bottom=2cm,left=3cm,right=3cm,marginparwidth=1.75cm]{geometry}
\usepackage{amsmath,amssymb,amsthm, float}
\usepackage{tikz}
\usepackage{physics}
\usepackage{graphicx}
\usepackage[colorlinks=true, allcolors=blue]{hyperref}
\usepackage{longtable,lipsum} 


\newtheorem{theorem}{Theorem}
\newtheorem{lemma}[theorem]{Lemma} % Lemmas share numbering with theorems
\newtheorem{corollary}[theorem]{Corollary}
\theoremstyle{definition}
\newtheorem{definition}[theorem]{Definition}
\newtheorem{example}[theorem]{Example}


\title{Simplicial Geometry}
\author{Radim Čech}

\begin{document}
\maketitle

\noindent Ahoj, já mám malý penis. 

\begin{definition}[Simplectic manifold]
    Let $M$ be a smooth manifold of even dimension $2m$ and let $\omega \in \Omega^2(M)$ be a closed non degenerate 2-form i.e.
    \begin{equation*}
        d\omega = 0 \text{ and } \omega^m = \omega \wedge \omega \wedge \dots \wedge \omega \not = 0,
    \end{equation*}
    Then $\omega$ is called a \textit{simplectic form} and the pair $(M, \omega)$ is called a \textit{simplectic manifold}.
\end{definition}

ekvivalentni definice nedegenerovanosti.

Narozdil od riemannovske geometrie nelze pouzit partitions of unity na konstrukci metriky.


\begin{example}[Canonical symplectic structure]
    Let $M = \mathbb{R}^2m$ with the global coordinates ${q_1, \dots, q_m, p_1, \dots, p_m}$. and let $\omega$ be a form s.t., 
    \begin{equation*}
        \omega = \sum_{i=1}^m dp_i \wedge dq_i.
    \end{equation*}
    Then 
    \begin{equation*}
        \omega^m = m! \cdot (-1)^{m(m-1)/2} \cdot dp_1 \wedge \dots \wedge dp_m \wedge dq_1 \wedge \dots \wedge dq_m
    \end{equation*}
    We call $R^2m$ with the form $\omega$ the canonical symplectic structure.
\end{example}

\begin{example}[Cotangent bundle is a symplectic manifold.]
    Let $M$ be a manifold of dimension $m$, let $\eta \in T^*M$ be a tangent covector and $\nu \in T_\eta(T^*M)$ be a tangent vector at $\eta$.
    Represent $\nu$ as a curve $\nu:(-\epsilon, \epsilon) \rightarrow T^*M$ s.t. 
    \begin{equation*}
        \nu(0) = \eta,\,\,\, \dot{\nu}(0) = \nu
    \end{equation*}
    Project this curve by the projection $\pi:T^*M \rightarrow M$ and apply $\eta$ to the tangent vector of the projected curve 
    \begin{equation}\label{eqLiouvilleForm}
        \theta(\nu) := \eta \Big(\frac{d}{dt}\big(\pi \circ \nu(t)\vert_{t=0} \big)\Big)
    \end{equation}
    Then 
    \begin{equation*}
        \omega = d \theta
    \end{equation*}
    is a simlectic form on $T^*M$.
    Any system of coordinates $\{q_1,\dots, q_m\}$ in M determines coordinates $\{q_1,\dots, q_m, p_1, \dots, p_m\}$ in $T^*M$ by the relation
    $\eta = \sum p_i \cdot dq_i$. From \eqref{eqLiouvilleForm} we have 
    
    
    TOHLE SPOCITAT!!!!!!!!!!!!!!
    
    \begin{equation*}
        \theta = \sum_{i=1}^m p_i \cdot dq_i
    \end{equation*}
    and the 2-form
    \begin{equation*}
        \omega = d \theta = \sum_{i=1}^m p_i \wedge dq_i
    \end{equation*}
    is non-degenerate.


\end{example}

\begin{lemma}[Lemmatko]
$2+2 = 4 - 1 = 3$ quick maffs.
\end{lemma}

A ted si rekneme dulezitou vetu.
\begin{theorem}[Hlavni veta o gaystvi]
Jsi gay.
\end{theorem}


\begin{corollary}
    Vlastne dusledek tohoto kratkeho textu je, ze bych se mel jit zabit. Jdu na to!
\end{corollary}

\end{document}