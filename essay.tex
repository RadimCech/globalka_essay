\documentclass{article}
\usepackage[english]{babel}
\usepackage[letterpaper,top=2cm,bottom=2cm,left=3cm,right=3cm,marginparwidth=1.75cm]{geometry}
\usepackage{amsmath,amssymb,amsthm, float}
\usepackage{tikz}
\usepackage{physics}
\usepackage{graphicx}
\usepackage[colorlinks=true, allcolors=blue]{hyperref}
\usepackage{longtable,lipsum} 


\newtheorem{theorem}{Theorem}
\newtheorem{lemma}[theorem]{Lemma} % Lemmas share numbering with theorems
\newtheorem{corollary}[theorem]{Corollary}
\theoremstyle{definition}
\newtheorem{definition}[theorem]{Definition}
\newtheorem{example}[theorem]{Example}


\title{Simplicial Geometry}
\author{Radim Čech}

\begin{document}
\maketitle

\noindent Ahoj, já mám malý penis. 

\begin{definition}[Simplectic manifold]
    Let $M$ be a smooth manifold of even dimension $2m$ and let $\omega \in \Omega^2(M)$ be a closed non degenerate 2-form i.e.
    \begin{equation*}
        d\omega = 0 \text{ and } \omega^m = \omega \wedge \omega \wedge \dots \wedge \omega \not = 0,
    \end{equation*}
    Then $\omega$ is called a \textit{simplectic form} and the pair $(M, \omega)$ is called a \textit{simplectic manifold}.
\end{definition}

ekvivalentni definice nedegenerovanosti.

Narozdil od riemannovske geometrie nelze pouzit partitions of unity na konstrukci metriky.

napsat poznamku o koncenci se psanim dimenze manifoldu :D


\begin{example}[Canonical symplectic structure]
    Let $M = \mathbb{R}^2m$ with the global coordinates ${q_1, \dots, q_m, p_1, \dots, p_m}$. and let $\omega$ be a form s.t., 
    \begin{equation*}
        \omega = \sum_{i=1}^m dp_i \wedge dq_i.
    \end{equation*}
    Then 
    \begin{equation*}
        \omega^m = m! \cdot (-1)^{m(m-1)/2} \cdot dp_1 \wedge \dots \wedge dp_m \wedge dq_1 \wedge \dots \wedge dq_m
    \end{equation*}
    We call $R^2m$ with the form $\omega$ the canonical symplectic structure.
\end{example}

\begin{example}[Cotangent bundle is a symplectic manifold.]
    Let $M$ be a manifold of dimension $m$, let $\eta \in T^*M$ be a tangent covector and $\nu \in T_\eta(T^*M)$ be a tangent vector at $\eta$. Represent $\nu$ as a curve $\nu:(-\epsilon, \epsilon) \rightarrow T^*M$ s.t. 
    \begin{equation*}
        \nu(0) = \eta,\,\,\, \nu'(0) = \nu
    \end{equation*}
    Project this curve by the projection $\pi:T^*M \rightarrow M$ and apply $\eta$ to the tangent vector of the projected curve 
    \begin{equation}\label{eqLiouvilleForm}
        \theta(\nu) := \eta \Big(\frac{d}{dt}\big(\pi \circ \nu(t)\vert_{t=0} \big)\Big)
    \end{equation}
    Then $\omega = d \theta$ is a simlectic form on $T^*M$. Any system of coordinates $\{q_1,\dots, q_m\}$ in M determines coordinates $\{q_1,\dots, q_m, p_1, \dots, p_m\}$ in $T^*M$ by the relation $\eta = \sum p_i \cdot dq_i$. From \eqref{eqLiouvilleForm} we have 
    
    
    TOHLE SPOCITAT!!!!!!!!!!!!!!
    
    \begin{equation*}
        \theta = \sum_{i=1}^m p_i \cdot dq_i
    \end{equation*}
    and the 2-form
    \begin{equation*}
        \omega = d \theta = \sum_{i=1}^m p_i \wedge dq_i
    \end{equation*}
    is non-degenerate.
\end{example}

\begin{theorem}[Darboux theorem]
    Let $M^{2m}, \omega$ be a symplectic manifold. Then for all $x \in M$ exists a chart $(U,\phi)$ such that $\omega$ is the pullback of the usual symplectic form $\omega = \phi*\Big(\sum_{m}^{i=1} dp_i \cdot dq_i \Big)$ 
\end{theorem}
\begin{proof}
    In $T^*_xM$ choose a basis $\sigma_1, \dots, \sigma_m, \mu_1, \dots, \mu_m$ such that $\omega$ is in normal form i.e.
    \begin{equation*}
        \omega(x) = \sum_{m}^{i=1} \sigma_i \wedge \mu_i.
    \end{equation*} 
    Now consider a chart $\phi : V \rightarrow \mathbb{R}^2m$ around $x$ such that 
    \begin{equation*}
        \phi(x) = 0 \text{ and } \omega(x) = \phi* \Big( \sum_{m}^{i=1} \sigma_i \wedge \mu_i(0) \Big)
    \end{equation*}
    Denote the corresponding symplectic form on $V$ by 
    \begin{equation*}
        \omega_1 = \phi* \Big( \sum_{m}^{i=1} \sigma_i \wedge \mu_i(0) \Big).
    \end{equation*}
    Then there exists a neighbourhood $U \subset V$ such that for all $t \in [0,1]$ the form 
    \begin{equation*}
        \omega_t = (1-t)omega + t\cdot omega_1
    \end{equation*}
    is a symplectic form on $U$. Also $d\omega_t = 0$ and since $\omega_t(x) = \omega(x)$ for all $t$. By compactness, all $\omega_t$ do not degenerate at the same time in a neighbourhood of $x$. 

    TADY POINCARE LEMMA 

    Since $d(w_1 - w_0) = 0$, Poincare lemma shows the existence of a 1-form $\alpha$ such that $\omega_1 - \omega_0 = d\alpha$.
    By substracting locally (if necessary) a 1-form with constant coefficients from $\alpha$, we may assume, that $\alpha$ vanishes at the point $x$, $\alpha(x) = 0$.
    By dualizing $\alpha$ by using the symplectic forms $\omega_t$, we get a family $\Omega_t$ of vector fields on $U$ parametrized by $t$
    \begin{equation*}
        \omega_t(\nu, \mathcal{W}_t) = \alpha(\nu).
    \end{equation*}
    Let $\varphi(y,t) \in M$ be a solution of the non-autonomous differential equation
    \begin{equation*}
        \varphi'(t) = \mathcal{W}_t(\varphi(t)), \hspace{5pt} \varphi(0) = y
    \end{equation*}

    All the vector fields $\mathcal{W}_t(x) = 0$ vanish at $x$, so the solution corresponding to $x$ is constant $\varphi(x,t) = x$. So there exists a neighbourhood of x $U_1 \in U$ such that for every initial condition $y \in U_1, \varphi(y,t)$ is defined on $[0,1]$. Let $\varphi_t:U_1 \rightarrow M$ be the corresponding map. Then the formula for 
    TADY CHYBI NEJAKA VETA :d
    implies that 



    \begin{equation*}
        \begin{split}
            \frac{d}{dt} \varphi_t^* (\omega_t) = \varphi_t^* \left( \frac{d\omega_t}{dt} \right) 
            + \varphi_t^* \left( \mathcal{L}_{\partial \varphi_t / \partial t} (\omega_t) \right)
            &= \varphi_t^* (\omega_t - \omega) + \varphi_t^* \left( \mathcal{L}_{W_t} (\omega_t) \right) \\ 
            = \varphi_t^* (\omega_t - \omega + d(W_t \lrcorner \omega_t) + W_t \lrcorner d\omega_t) 
            &= \varphi_t^* (\omega_t - \omega - d\alpha) = 0.
        \end{split}
    \end{equation*}



    Thus $(\varphi_t^*(\omega_1) = \varphi_0^*(\omega) = \omega)$, and $(\Phi \circ \varphi_1)$ is the chart we were looking for,

    \begin{equation*}
        (\Phi \circ \varphi_1)^* \left( \sum_{i=1}^m dp_i \wedge dy_i \right) = \omega.
    \end{equation*}

\end{proof}

\begin{lemma}[Lemmatko]
$2+2 = 4 - 1 = 3$ quick maffs.
\end{lemma}

A ted si rekneme dulezitou vetu.
\begin{theorem}[Hlavni veta o gaystvi]
Jsi gay.
\end{theorem}


\begin{corollary}
    Vlastne dusledek tohoto kratkeho textu je, ze bych se mel jit zabit. Jdu na to!
\end{corollary}

\end{document}